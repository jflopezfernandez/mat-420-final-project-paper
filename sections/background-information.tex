\section{Background Information}
Canada geese may have forever cemented their legacy of infamy in 2009,  when a flock of geese flying  too  close  to  a  commercial  airliner  were sucked  into  the  turbines,  immediately  disabling the  aircraft  in  mid-air. Captain  ’Sully’  Sullenberger and his First Officer Jeff Skiles were able to safely land the plane in the Hudson river, saving the lives of all 155 passengers and crew (Service, 2019).  This  was  not  the  first  incident  of  its  kind, nor  was  it  the  last,  unfortunately.   In  fact,  these incidents are poised to become much more common,  as the buffer zones airports have created to prevent these very incidents are greatly coveted by the same pests they are meant to repel. According to University of Illinois graduate student researcher Ryan Askren, who lead a study that used GPS to track the movements of thousands of Canada geese, it seems Canada geese are flocking to these buffer zones precisely because the buffer zones are much emptier and quieter than the parks and recreational areas in which most people are accustomed  to  finding  geese  (Yates,  2019).   While most  of  the  geese tracked in the study  did  not  cause  any  problems, nearly a third of them came within 10,000 feet of a runway, with almost 400 geese even crossing an aircraft flight  path.  Luckily, there were no incidents.This situation is not sustainable. If there is anything to be learned from tragedies like the Elmendorf Air Force Base tragedy, it’s that the Hudson river landing was a lucky break, and in a less fortunate world, we would be mourning the loss of every single one of those people (CNN, 1995). When one considers the 1999 technical report by Dr. Paul Curtis of Cornell University in which he states that"most modern aircraft are engineered only to withstand the impact or engine ingestion of a single 1 to 3-pound bird (Curtis, 1999)," the lateness of the hour becomes clear; action must be taken.